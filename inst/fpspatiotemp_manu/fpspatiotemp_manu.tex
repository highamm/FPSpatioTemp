% interactcadsample.tex
% v1.03 - April 2017

\documentclass[]{interact}

\usepackage{epstopdf}% To incorporate .eps illustrations using PDFLaTeX, etc.
\usepackage{subfigure}% Support for small, `sub' figures and tables
%\usepackage[nolists,tablesfirst]{endfloat}% To `separate' figures and tables from text if required

\usepackage{natbib}% Citation support using natbib.sty
\bibpunct[, ]{(}{)}{;}{a}{}{,}% Citation support using natbib.sty
\renewcommand\bibfont{\fontsize{10}{12}\selectfont}% Bibliography support using natbib.sty

\theoremstyle{plain}% Theorem-like structures provided by amsthm.sty
\newtheorem{theorem}{Theorem}[section]
\newtheorem{lemma}[theorem]{Lemma}
\newtheorem{corollary}[theorem]{Corollary}
\newtheorem{proposition}[theorem]{Proposition}

\theoremstyle{definition}
\newtheorem{definition}[theorem]{Definition}
\newtheorem{example}[theorem]{Example}

\theoremstyle{remark}
\newtheorem{remark}{Remark}
\newtheorem{notation}{Notation}

% see https://stackoverflow.com/a/47122900

% Pandoc citation processing

\usepackage{hyperref}
\usepackage[utf8]{inputenc}
\def\tightlist{}


\begin{document}

\articletype{ARTICLE TEMPLATE}

\title{An Application of Spatiotemporal Modeling to Finite Population
Abundance Prediction}


\author{\name{Matt Higham$^{a}$, Carly Hammond, John Merickel, Jeff
Wells$^{b}$, add more$^{c, \dagger, \ddagger}$}
\affil{$^{a}$St.~Lawrence University Canton, NY 13617; $^{b}$fill in
later; $^{c}$fill in later}
}

\thanks{CONTACT Matt
Higham. Email: \href{mailto:mhigham@stlawu.edu}{\nolinkurl{mhigham@stlawu.edu}}, Carly
Hammond, John Merickel, Jeff Wells. Email: fill these in later, add
more. Email: \href{mailto:leutnant@fh-muenster.de}{\nolinkurl{leutnant@fh-muenster.de}}}

\maketitle

\begin{abstract}
Insert abstract here.
\end{abstract}

\begin{keywords}
spatial; temporal; kriging;
\end{keywords}

\section{Introduction}

\subsection{Motivation}

Moose surveys in Alaska and western Canada are often performed annually
in many regions. The primary goal of these surveys is to predict moose
abundance, the total number of moose, in the region. Because of time and
money constraints, only some areas (sites) in the region of interest are
selected to be in the survey. Biologists fly to these selected sites,
count the number of moose, and can then use a spatial statistical model
to find a prediction for the finite abundance for that year
\citep{ver2008spatial}.

Though these surveys are annual, each survey is analysed completely
independently of surveys from previous years
\citep[e.g.][]{gasaway1986estimating, kellie_geospatial_2006, boertje2009managing, peters2014contrasting}.
For example, a model for a survey conducted in the year 2019 only uses
counts on sites that were sampled in that year. However, using counts
from previous years in a model that incorporates both spatial and
temporal correlation (spatiotemporal) could result in a prediction that
is more precise than predictions from a spatial model using only counts
from the most recent survey year.

Though the framework of the motivation is given with an example on moose
surveys, this type of analysis could be useful for many examples
involving prediction in a finite region with spatial sites that are
surveyed regularly.

\subsection{Background}

\begin{itemize}
\tightlist
\item
  Add paragraph about background of spatiotemporal models
\end{itemize}

Prediction for a total, a subset of the total, or a mean in a finite
number of spatial locations should incorporate a finite population
correction to the variance of the predictor
\citep{ver2008spatial, higham2021adjusting}. In the context of
ecological monitoring in spatiotemporal prediction, we are often most
interested in predicting the total abundance for the most recent year of
the survey. In this case, the finite population correction should adjust
based on the number of sites surveyed in the current year of the survey,
so that, for example, the prediction variance is zero if all sites in
the current year are sampled.

The rest of this paper is organized as follows. In Section
\ref{section:Methods}, we couple spatiotemporal modeling with finite
population prediction to develop the Best-Linear-Unbiased-Predictor for
any linear function of site abundance, including the total abundance
across all sites. In Section \ref{section:Application}, we apply the
predictor to a moose data set in the TOC region of Alaska. In Section
\ref{section:Simulation}, we conduct a brief simulation study to examine
the properties of the predictor. Finally, in Section
\ref{section:Discussion}, we conclude and give directions for future
research.

\section{Methods} \label{section:Methods}

We now give details on the development of the predictor for abundance.
We first detail the spatiotemporal model, and we then use the
spatiotemporal model with a finite population correction factor to give
a Best-Linear-Unbiased-Predictor (BLUP) and its variance for total
abundance in a given year.

\subsection{Spatiotemporal Model}

Let \(Y(\mathbf{s}_{i}, t_j)\), \(i = 1, 2, \ldots, n_{sp}\) and
\(j = 1, 2, \ldots, n_{t}\) be a random variable, where the vector
\(\mathbf{s}_i\) contains the coordinates for the \(i^{th}\) spatial
site location and where \(t_j\) is the \(j^{th}\) time point. With each
spatial location at each time point, the total number of data points
(observed and unobserved) is \(n_{sp} \cdot n_{t} \equiv N\). Then, a
model for \(\mathbf{y}(\mathbf{s}_{i}, t_j)\), a vector of the
\(Y(\mathbf{s}_{i}, t_j)\) is

\begin{equation}
\mathbf{y}(\mathbf{s}_{i}, t_j) = \mathbf{X} \bm{\beta} + \bm{\epsilon}(\mathbf{s}_{i}, t_j),
\end{equation} \noindent where \(\mathbf{X}\) is a design matrix for
fixed effects and \(\bm{\beta}\) is a parameter vector of fixed effects.
The error \(\bm{\epsilon}(\mathbf{s}_{i}, t_j)\) can be decomposed into
spatial and temporal components, as in \citet{dumelle2021linear}. In
particular, a sum-with-error linear mixed model for response vector
\(\mathbf{y}(\mathbf{s}_{i}, t_j)\) is

\begin{equation} \label{equation:model}
\mathbf{y}(\mathbf{s}_{i}, t_j) = \mathbf{X} \bm{\beta} + \mathbf{Z}_{sp} \bm{\delta} + \mathbf{Z}_{sp} \bm{\gamma} + \mathbf{Z}_t \bm{\tau} + \mathbf{Z}_t \bm{\eta} + \bm{\nu}.
\end{equation}

\(\mathbf{Z}_{sp}\) is an \(N \times n_{sp}\) matrix where the values in
a row corresponding to an observation in location \(\mathbf{s}_{i}\) are
a \(1\) in the \(i^{th}\) column and 0's in all other columns.
Similarly, \(\mathbf{Z}_t\) is an \(N \times n_{t}\) matrix where the
values in a row corresponding to an observation at time point
\(\mathbf{t}_j\) are a \(1\) in the \(j^{th}\) column and 0's in all
other columns. We assume that \(\bm{\delta}\), \(\bm{\gamma}\),
\(\bm{\tau}\), \(\bm{eta}\), and \(\bm{\nu}\) all are all mean
\(\mathbf{0}\) vectors of length \(n_{sp}\), \(n_{sp}\), \(n_t\),
\(n_t\), and \(N\), respectively, with the covariances \mbox{}
\begin{align*}
cov(\bm{\delta}) &= \sigma^2_{\delta} \mathbf{R}_{sp} \\
cov(\bm{\gamma}) &= \sigma^2_{\gamma} \mathbf{I}_{sp} \\
cov(\bm{\tau}) &= \sigma^2_{\tau} \mathbf{R}_t \\
cov(\bm{\eta}) &= \sigma^2_{\eta} \mathbf{I}_t \\
cov(\bm{\nu}) &= \sigma^2_{\nu} \mathbf{I}_N,
\end{align*} \noindent where \(\mathbf{R}_{sp}\) is a spatial
correlation matrix and \(\mathbf{R}_t\) is a temporal correlation
matrix. \(\sigma^2_{\delta}\) and \(\sigma^2_{\gamma}\) are the spatial
partial sill and spatial nugget, \(\sigma^2_{\tau}\) and
\(\sigma^2_{\eta}\) are the temporal partial sill and temporal nugget,
and \(\sigma^2_{\nu}\) is spatiotemporal independent error variance
parameter.

If we assume that \(\bm{\delta}\), \(\bm{\gamma}\), \(\bm{\tau}\),
\(\bm{\eta}\), and \(\bm{\nu}\) are mutually independent of each other,
then

\begin{equation}
var(\mathbf{y}) \equiv \bm{\Sigma} = \sigma^2_{\delta} \mathbf{Z}_{sp} \mathbf{R}_{sp} \mathbf{Z}_{sp}' + \sigma^2_{\gamma} \mathbf{Z}_{sp} \mathbf{I}_{sp} \mathbf{Z}_{sp}' + \sigma^2_{\tau} \mathbf{Z}_t \mathbf{R}_t \mathbf{Z}_t'+ \sigma^2_{\eta} \mathbf{Z}_t \mathbf{I}_t \mathbf{Z}_t' + \sigma^2_{\nu} \mathbf{I}_N.
\end{equation}

There are many common parameterizations of \(\mathbf{R}_{sp}\), but one
form is the exponential. For observations at locations \(i\) and \(i'\)
at \(h_{ii'}\) distance apart, row \(i\) and column \(i'\) of
\(\mathbf{R}_{sp}\) is equal to \mbox{} \begin{equation}
\text{exp}(-h_{ii'} / \phi),
\end{equation} \noindent where \(\phi\) is the range parameter.

There are also many common parameterizations of \(\mathbf{R}_t\), but
one form is the exponential, which is equivalent to an AR(1) process in
time series if the time points are equally spaced and the correlation
parameter is greater than zero. For observations at time points \(j\)
and \(j'\) that are \(m_{jj'}\) units apart, row \(j\) and column \(j'\)
of \(\mathbf{R}_{t}\) is equal to \mbox{} \begin{equation}
\text{exp}(-m_{jj'} / \rho),
\end{equation} \noindent where \(\rho\) is the autocorrelation
parameter.

If we assume that \(\mathbf{y}\) is multivariate normal with mean
\(\mathbf{X} \bm{\beta} \equiv \bm{\mu}\) and variance \(\bm{\Sigma}\),
then all parameters can be estimated with Maximum Likelihood or
Restricted Maximum Likelihood.

Note that we will explore other covariance structures, including an
extension of the sum-with-error model and different spatial and temporal
correlation structures.

\subsection{Finite Population Kriging}

The model in equation \ref{equation:model} is used for all \(N\)
observations at \(n_{sp}\) sites and \(n_t\) time points. But, we
typically do not have the resources to sample every spatial site in
every year.

Let the subscript \(s\) denote observations that were sampled (both past
and present), and let the subscript \(u\) denote observations that were
unsampled. Then, we can re-order the response vector so that \mbox{}
\begin{equation}
\mathbf{y} = [\mathbf{y}_u', \mathbf{y}_s']'.
\end{equation}

Let
\(\mathbf{\tilde{y}} = [\mathbf{\tilde{y}}_u', \mathbf{\tilde{y}}_s']'\)
denote the fixed, realized values of the response variable for one
data-generating process. Our primary goal is to use the model developed
in the previous section to predict values for \(\mathbf{\tilde{y}}_{u}\)
from the observed data in \(\mathbf{\tilde{y}}_{s}\). That is, we want
to find optimal weights \(\mathbf{a}'\) to apply to the sampled data
\(\mathbf{a}' \mathbf{\tilde{y}}_s\), such that
\(\mathbf{a}' \mathbf{y}_s\) is the Best Linear Unbiased Predictor
(BLUP) for \(\mathbf{b}_a' \mathbf{y}_a\). If we are interested in the
total abundance across all years, then \(\mathbf{b}_a\) is a column
vector of 1's. so that we are adding up all values of the response for
our predictor of total abundance.

Unbiasedness implies that
\(E(\mathbf{a'}\mathbf{y}_s) = E(\mathbf{b}_a'\mathbf{y}_a)\) for all
\(\bm{\beta}\). So, denoting \(\mathbf{X}_s\) as the design matrix for
sampled sites, \(\mathbf{a'} \mathbf{X}_s \bm{\beta}\) =
\(\mathbf{b'} \mathbf{X} \bm{\beta}\) for every \(\bm{\beta}\), implying
that \(\mathbf{a'} \mathbf{X}_s = \mathbf{b'}_a \mathbf{X}\).

The kriging weights are then found by finding \(\bm{\lambda}\) such that
\mbox{]} \begin{equation}
E\{(\mathbf{a'}\mathbf{y}_s - \mathbf{b'}_a \mathbf{y}_a)(\mathbf{a'}\mathbf{y}_s - \mathbf{b'}_a \mathbf{z}_a)\} - E\{(\bm{\lambda'}\mathbf{y}_s - \mathbf{b'}_a \mathbf{z}_a)(\bm{\lambda}'\mathbf{y}_s - \mathbf{b'}_a \mathbf{z}_a)\}
\end{equation} \noindent is greater than 0 for all \(\mathbf{a'}\). The
prediction equations are

\begin{equation}
\begin{pmatrix}
\bm{\Sigma}_{s, s} & \mathbf{X}_s \\
\mathbf{X}_s' & 0
\end{pmatrix} 
\begin{pmatrix}
\bm{\lambda} \\
m
\end{pmatrix} = 
\begin{pmatrix}
\bm{\Sigma}_{s, s} & \bm{\Sigma}_{s, u} \\
\mathbf{X}_{s}' & \mathbf{X}_{u}'
\end{pmatrix} 
\begin{pmatrix}
\mathbf{b}_{s} \\
\mathbf{b}_{u}
\end{pmatrix},
\end{equation} \noindent where again the subscripts \(s\) and \(u\)
denote sampled and unsampled observations. For example, letting \(n\)
denote the number of sampled observations, \(\bm{\Sigma}_{s, s}\)
denotes the \(n \times n\) submatrix of \(\bm{\Sigma}\) corresponding
only to rows and columns of sampled observations and
\(\bm{\Sigma}_{u, s}\) denotes the \((N - n) \times n\) submatrix of
\(\bm{\Sigma}\) corresponding to rows of observations that were not
sampled and columns of observations that were sampled. Then,
\(\bm{\lambda}\) is an \(n \times 1\) vector.

Then, we can solve for the prediction weights as \mbox{}
\begin{equation}
\bm{\lambda}_s = \mathbf{b}_{s}' + \mathbf{b}_{u}' (\bm{\Sigma}_{u, s}\bm{\Sigma}_{s, s}^{-1}) - \mathbf{b}'_{u}(\bm{\Sigma}_{u, s} \bm{\Sigma}_{s, s}^{-1})\mathbf{X}_s(\mathbf{X}_s'\bm{\Sigma}_{s, s}^{-1}\mathbf{X}_s)^{-1}\mathbf{X}_s'\bm{\Sigma}_{s, s}^{-1} + \mathbf{b}_{u}' \mathbf{X}_{u}'(\mathbf{X}_s'\bm{\Sigma}_{s, s}^{-1}\mathbf{X}_s)^{-1}\mathbf{X}_s \bm{\Sigma}_{s, s}^{-1}.
\end{equation} \noindent Our prediction for the total abundance across
all years of the survey is then \mbox{} \begin{equation}
\bm{\lambda}_s' \mathbf{\tilde{y}}_s,
\end{equation} \noindent with a prediction variance of \mbox{}
\begin{equation}
E((\bm{\lambda}_s'\mathbf{y}_s - \mathbf{b}_a'\mathbf{y}_a)(\bm{\lambda}_s'\mathbf{y}_s - \mathbf{b}_a'\mathbf{y}_a)) = \\
\bm{\lambda}_s'\bm{\Sigma}_{s, s}\bm{\lambda}_s - 2 \mathbf{b}_a' \bm{\Sigma}_{a ,s} \bm{\lambda}_s + \mathbf{b}_a' \bm{\Sigma}\mathbf{b}_a.
\end{equation}

However, we often are not interested in predicting total abundance
across multiple years and instead would want a prediction of the total
abundance in the most recent year of the survey. Therefore, for this
goal, \(\mathbf{b}_a\) should not be a vector of 1's and should instead
take a value of 1 if the observation is in the current year of the
survey and a 0 if the observation is not in the current year of the
survey: \mbox{} \begin{equation}
\mathbf{b}_a = [\mathbf{b}_{up}^\prime, \mathbf{b}_{uc}', \mathbf{b}_{sp}', , \mathbf{b}_{sc}']' = [\mathbf{0}', \mathbf{1}', \mathbf{0}', \mathbf{1}']',
\end{equation} \noindent where the subscripts \(up\), \(uc\), \(sp\),
and \(sc\) denote unsampled sites in past years, unsampled sites in
current years, sampled sites in past years, and sampled sites in current
years, respectively.

\(\bm{\lambda}_s\) can then be rewritten as \mbox{} \begin{equation}
\bm{\lambda}_s = \mathbf{b}_{s}' + \mathbf{b}_{uc}' (\bm{\Sigma}_{uc, s}\bm{\Sigma}_{s, s}^{-1}) - \mathbf{b}'_{uc}(\bm{\Sigma}_{uc, s} \bm{\Sigma}_{s, s}^{-1})\mathbf{X}_s(\mathbf{X}_s'\bm{\Sigma}_{s, s}^{-1}\mathbf{X}_s)^{-1}\mathbf{X}_s'\bm{\Sigma}_{s, s}^{-1} + \mathbf{b}_{uc}' \mathbf{X}_{uc}'(\mathbf{X}_s'\bm{\Sigma}_{s, s}^{-1}\mathbf{X}_s)^{-1}\mathbf{X}_s \bm{\Sigma}_{s, s}^{-1}.
\end{equation}

The prediction variance can then be rewritten as \mbox{}
\begin{equation}
\bm{\lambda}_s'\bm{\Sigma}_{s, s}\bm{\lambda}_s - 2 \mathbf{b}_{c}' \bm{\Sigma}_{c, s} \bm{\lambda}_s + \mathbf{b}_{c}' \bm{\Sigma_{c, c}} \mathbf{b}_{c},
\end{equation} \noindent where \(c\) denotes observations in the current
year.

\section{Application} \label{section:Application}

Abundance surveys are performed in the TOK region of Alaska annually. In
particular, surveys were conducted every year from 2014 through 2020,
except for the year 2016, during which there was not sufficient snow
cover to perform a survey. Before each survey, the sites are stratified
into a High stratum and a Low stratum. In the high stratum, there are
230 unique spatial locations in the sampling frame of each year while in
the low stratum, there are 151 unique spatial locations. For both
strata, there are 7 unique time points, including the year 2016.

The strata are fit separately with a sum-with-error covariance, an
exponential spatial correlation structure, and an AR(1) temporal
correlation structure outlined in \ref{section:Methods}. The predicted
total abundance is 2811 moose with a 95\% prediction interval of (2683,
2940) moose.

Moose surveys in the TOC region were historically analyzed without
explicitly using any data from surveys in past years. Therefore, we
compare the spatiotemporal prediction and prediction interval with a
spatial model fit with the \(\texttt{sptotal}\) package using only the
data from the year 2020. The prediction total abundance is 2888 moose
with a 95\% prediction interval of (2306, 3469) moose. We see that the
predictions are somewhat similar, but that, because the strictly spatial
analysis ignores information from past years, the prediction interval
for the spatiotemporal analysis is more narrow.

\section{Simulation} \label{section:Simulation}

\begin{itemize}
\tightlist
\item
  possibly include \texttt{sptotal} (on current year only) and SRS (on
  current year only) as reference comparisons.
\end{itemize}

For a preliminary simulation, we use a grid of 100 spatial sites and 5
time points. The spatiotemporal process is simulated as a sum-with-error
model with an exponential spatial correlation structure and an
exponential temporal correlation structure with the following parameters

The mean is

\begin{itemize}
\tightlist
\item
  \(\beta\) = 10.
\end{itemize}

The spatial parameters are

\begin{itemize}
\tightlist
\item
  \(\sigma^2_{\delta}\) = 0.9,
\item
  \(\sigma^2_{\gamma}\) = 0.1,
\item
  \(\phi\) = 5.
\end{itemize}

The temporal parameters are

\begin{itemize}
\tightlist
\item
  \(\sigma^2_{\tau}\) = 0.7,
\item
  \(\sigma^2_{\eta}\) = 0.3,
\item
  \(\rho\) = 0.8.
\end{itemize}

And, the spatiotemporal nugget is

\begin{itemize}
\tightlist
\item
  \(\sigma^2_{\nu}\) = 0.4.
\end{itemize}

The sample size \(n\) is 100 (of the 500 total data points). For 100
iterations, the percentage of 90\% prediction intervals that covered the
true total was 81\%.

There are a few plausible reasons for this low coverage:

\begin{itemize}
\tightlist
\item
  there is something incorrect about the method or code used.
\item
  the sample size is only 100 sites, meaning that, on average, only 20
  sites get selected per year. This small sample size may mean that the
  8 parameters cannot be estimated accurately. I believe that this is
  the actual cause of the lower than nominal coverage and hope to do a
  larger simulation study next.
\item
  only 100 simulations were done.
\end{itemize}

Hampering investigation of this is the fact that the simulations take a
long time to run. The \texttt{stmodel} package will be very helpful for
this, as we can replace the slow code for fitting the spatiotemporal
model with faster code from the package.

\section{Discussion} \label{section:Discussion}

\begin{itemize}
\item
  mention substantial reduction of se in the application (and,
  presumably, the simulations).
\item
  mention normal-based-related limitations
\item
  mention Bayesian approach, and its potential flaws
\item
  mention possible extension to imperfect detection
\item
  mention forecasting potential
\item
  take-home message: monitoring programs that use regularly surveys
  might consider incorporating time into their analysis to improve
  precision of predictors (e.g.~NARS for lake assessments).
\end{itemize}

\bibliographystyle{tfcad}
\bibliography{interactcadsample.bib}




\end{document}

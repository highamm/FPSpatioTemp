% interactcadsample.tex
% v1.03 - April 2017

\documentclass[]{interact}

\usepackage{epstopdf}% To incorporate .eps illustrations using PDFLaTeX, etc.
\usepackage{subfigure}% Support for small, `sub' figures and tables
%\usepackage[nolists,tablesfirst]{endfloat}% To `separate' figures and tables from text if required

\usepackage{natbib}% Citation support using natbib.sty
\bibpunct[, ]{(}{)}{;}{a}{}{,}% Citation support using natbib.sty
\renewcommand\bibfont{\fontsize{10}{12}\selectfont}% Bibliography support using natbib.sty

\theoremstyle{plain}% Theorem-like structures provided by amsthm.sty
\newtheorem{theorem}{Theorem}[section]
\newtheorem{lemma}[theorem]{Lemma}
\newtheorem{corollary}[theorem]{Corollary}
\newtheorem{proposition}[theorem]{Proposition}

\theoremstyle{definition}
\newtheorem{definition}[theorem]{Definition}
\newtheorem{example}[theorem]{Example}

\theoremstyle{remark}
\newtheorem{remark}{Remark}
\newtheorem{notation}{Notation}

% see https://stackoverflow.com/a/47122900

% Pandoc citation processing

\usepackage{hyperref}
\usepackage[utf8]{inputenc}
\def\tightlist{}


\begin{document}

\articletype{ARTICLE TEMPLATE}

\title{An Application of Spatiotemporal Modeling to Finite Population
Abundance Prediction}


\author{\name{A. N. Author$^{a}$, John Smith$^{b}$, Dominik
Leutnant$^{c, \dagger, \ddagger}$}
\affil{$^{a}$Taylor \& Francis, 4 Park Square, Milton Park, Abingdon,
UK; $^{b}$Institut für Informatik, Albert-Ludwigs-Universität, Freiburg,
Germany; $^{c}$Muenster University of Applied Sciences - Institute for
Infrastructure, Water, Resources, Environment, Correnstr. 25, 48149
Münster, Germany}
}

\thanks{CONTACT A. N.
Author. Email: \href{mailto:latex.helpdesk@tandf.co.uk}{\nolinkurl{latex.helpdesk@tandf.co.uk}}, John
Smith. Email: \href{mailto:john.smith@uni-freiburg.de}{\nolinkurl{john.smith@uni-freiburg.de}}, Dominik
Leutnant. Email: \href{mailto:leutnant@fh-muenster.de}{\nolinkurl{leutnant@fh-muenster.de}}}

\maketitle

\begin{abstract}
This template is for authors who are preparing a manuscript for a Taylor
\& Francis journal using the \LaTeX~document preparation system and the
`interact\} class file, which is available via selected journals' home
pages on the Taylor \& Francis website.
\end{abstract}

\begin{keywords}
Sections; lists; figures; tables; mathematics; fonts; references;
appendices
\end{keywords}

\section{Introduction}

\subsection{Motivation}

Moose surveys in Alaska and western Canada are often performed annually
in many regions. The primary goal of these surveys is to predict moose
abundance, the total number of moose, in the region. Because of time and
money constraints, only some areas (sites) in the region of interest are
selected to be in the survey. Biologists fly to these selected sites,
count the number of moose, and can then use a spatial statistical model
to find a prediction for the finite abundance for that year
\citep{ver2008spatial}.

Though these surveys are annual, each survey is analysed completely
independently of surveys from previous years (cite examples). For
example, a model for the survey conducted in the year 2019 only uses
counts on sites that were sampled in that year. However, using counts
from previous years in a model that incorporates both spatial and
temporal correlation (spatiotemporal) could result in a prediction that
is more precise than predictions from a spatial model using only counts
from the most recent survey year.

Though the framework of the motivation is given with an example on moose
surveys, this type of analysis could be useful for many examples
involving prediction in a finite region with sites that are surveyed
regularly.

\subsection{Background}

Paragraph about background of spatiotemporal models

Paragraph about background of finite populations

Small paragraph combining and summarizing the previous two paragraphs

Small paragraph that gives the outline for the rest of the paper

\section{Methods}

We now give details on the development of the predictor for abundance.
We first detail the spatiotemporal model, and we then use the
spatiotemporal model with a finite population correction factor to give
a Best-Linear-Unbiased-Predictor (BLUP) and its variance for total
abundance in a given year.

\subsection{Spatiotemporal Model}

Let \(Y(\mathbf{s}_{i}, t_j)\), \(i = 1, 2, \ldots, n_{sp}\) and
\(j = 1, 2, \ldots, n_{t}\) be a random variable, where the vector
\(\mathbf{s}_i\) contains the coordinates for the \(i^{th}\) spatial
site location and where \(t_j\) is the \(j^{th}\) time point. With each
spatial location at each time point, the total number of data points
(observed and unobserved) is \(n_{sp} \cdot n_{t} \equiv N\). Then, a
model for \(\\mathbf{y}(\mathbf{s}_{i}, t_j)\), a vector of the
\(Y(\mathbf{s}_{i}, t_j)\) is

\begin{equation}
\mathbf{y}(\mathbf{s}_{i}, t_j) = \mathbf{X} \bm{\beta} + \bm{\epsilon}(\mathbf{s}_{i}, t_j),
\end{equation} \noindent where \(\mathbf{X}\) is a design matrix for
fixed effects and \(\bm{\beta}\) is a parameter vector of fixed effects.
The error \(\bm{\epsilon}(\mathbf{s}_{i}, t_j)\) can be decomposed into
spatial and temporal components, as in \citet{dumelle2021linear}. In
particular, a sum-with-error linear mixed model for response vector
\(\mathbf{y}(\mathbf{s}_{i}, t_j)\) is

\begin{equation} \label{equation:model}
\mathbf{y}(\mathbf{s}_{i}, t_j) = \mathbf{X} \bm{\beta} + \mathbf{Z}_{sp} \bm{\delta} + \mathbf{Z}_{sp} \bm{\gamma} + \mathbf{Z}_t \bm{\tau} + \mathbf{Z}_t \bm{\eta} + \bm{\nu}.
\end{equation}

\(\mathbf{Z}_{sp}\) is an \(N \times n_{sp}\) matrix where the values in
a row corresponding to an observation in location \(\mathbf{s}_{i}\) are
a \(1\) in the \(i^{th}\) column and 0's in all other columns.
Similarly, \(\mathbf{Z}_t\) is an \(N \times n_{t}\) matrix where the
values in a row corresponding to an observation at time point
\(\mathbf{t}_j\) are a \(1\) in the \(j^{th}\) column and 0's in all
other columns. We assume that \(\bm{\delta}\), \(\bm{\gamma}\),
\(\bm{\tau}\), \(\bm{eta}\), and \(\bm{\nu}\) all are all mean
\(\mathbf{0}\) vectors of length \(n_{sp}\), \(n_{sp}\), \(n_t\),
\(n_t\), and \(N\), respectively, with the covariances \mbox{}
\begin{align*}
cov(\bm{\delta}) &= \sigma^2_{\delta} \mathbf{R}_{sp} \\
cov(\bm{\gamma}) &= \sigma^2_{\gamma} \mathbf{I}_{sp} \\
cov(\bm{\tau}) &= \sigma^2_{\tau} \mathbf{R}_t \\
cov(\bm{\eta}) &= \sigma^2_{\eta} \mathbf{I}_t \\
cov(\bm{\nu}) &= \sigma^2_{\nu} \mathbf{I}_N,
\end{align*} \noindent where \(\mathbf{R}_{sp}\) is a spatial
correlation matrix and \(\mathbf{R}_t\) is a temporal correlation
matrix. \(\sigma^2_{\delta}\) and \(\sigma^2_{\gamma}\) are the spatial
partial sill and spatial nugget, \(\sigma^2_{\tau}\) and
\(\sigma^2_{\eta}\) are the temporal partial sill and temporal nugget,
and \(\sigma^2_{\nu}\) is spatiotemporal independent error variance
parameter.

If we assume that \(\bm{\delta}\), \(\bm{\gamma}\), \(\bm{\tau}\),
\(\bm{\eta}\), and \(\bm{\nu}\) are mutually independent of each other,
then

\begin{equation}
var(\mathbf{y}) \equiv \bm{\Sigma} = \sigma^2_{\delta} \mathbf{Z}_{sp} \mathbf{R}_{sp} \mathbf{Z}_{sp}' + \sigma^2_{\gamma} \mathbf{Z}_{sp} \mathbf{I}_{sp} \mathbf{Z}_{sp}' + \sigma^2_{\tau} \mathbf{Z}_t \mathbf{R}_t \mathbf{Z}_t'+ \sigma^2_{\eta} \mathbf{Z}_t \mathbf{I}_t \mathbf{Z}_t' + \sigma^2_{\nu} \mathbf{I}_N.
\end{equation}

One common form of \(\mathbf{R}_{sp}\) is exponential. For observations
at locations \(i\) and \(i'\) at \(h_{ii'}\) distance apart, the
\(i^{th}\) row and \(i'^{th}\) column of \(\mathbf{R}_{sp}\) is \mbox{}
\begin{equation}
r_{sp, ii'} = \text{exp}(-h_{ii'} / \phi),
\end{equation} \noindent where \(\phi\) is the range parameter.

One common form of \(\mathbf{R}_t\) is also exponential, more commonly
known in time series as AR(1). For observations at time points \(j\) and
\(j'\) that are \(m_{jj'}\) units apart, the \(j^{th}\) row and
\(j'^{th}\) column of \(\mathbf{R}_{t}\) is \mbox{} \begin{equation}
r_{t, jj'} = \text{exp}(-m_{jj'} / \rho),
\end{equation} \noindent where \(\rho\) is the autocorrelation
parameter.

If we assume that \(\mathbf{y}\) is multivariate normal with mean
\(\mathbf{X} \bm{\beta} \equiv \bm{\mu}\) and variance \(\bm{\Sigma}\),
then all parameters can be estimated with Maximum Likelihood or
Restricted Maximum Likelihood.

\subsection{Finite Population Kriging}

The model in equation \ref{equation:model} is used for all \(N\)
observations at \(n_{sp}\) sites and \(n_t\) time points. But, we
typically do not have the resources to sample every spatial site in
every year. Additionally, for many wildlife management decisions, we are
most interested in prediction of the abundance in the most current year
of the survey.

Let the subscript \(s\) denote observations that were sampled (both past
and present), and let the subscript \(u\) denote observations that were
unsampled. Then, we can re-order the response vector so that \mbox{}
\begin{equation}
\mathbf{y} = [\mathbf{y}_u', \mathbf{y}_s']'.
\end{equation}

Let
\(\mathbf{\tilde{y}} = [\mathbf{\tilde{y}}_u', \mathbf{\tilde{y}}_s']'\)
denote the fixed, realized values of the response variable for one
data-generating process. Our primary goal is to use the model developed
in the previous section to predict values for \(\mathbf{\tilde{y}}_{u}\)
from the observed data in \(\mathbf{\tilde{y}}_{s}\). That is, we want
to find optimal weights \(\mathbf{a}'\) to apply to the sampled data
\(\mathbf{a}' \mathbf{\tilde{y}}_s\), such that
\(\mathbf{a}' \mathbf{y}_s\) is the Best Linear Unbiased Predictor
(BLUP) for \(\mathbf{b}_a' \mathbf{y}_a\). If we are interested in the
total abundance across all years, then \(\mathbf{b}_a\) is a column
vector of 1's. so that we are adding up all values of the response for
our predictor of total abundance.

Unbiasedness implies that
\(E(\mathbf{a'}\mathbf{y}_s) = E(\mathbf{b}_a'\mathbf{y}_a)\) for all
\(\bm{\beta}\). So, denoting \(\mathbf{X}_s\) as the design matrix for
sampled sites, \(\mathbf{a'} \mathbf{X}_s \bm{\beta}\) =
\(\mathbf{b'} \mathbf{X} \bm{\beta}\) for every \(\bm{\beta}\), implying
that \(\mathbf{a'} \mathbf{X}_s = \mathbf{b'}_a \mathbf{X}\).

The kriging weights are then found by finding \(\bm{\lambda}\) such that
\mbox{]} \begin{equation}
E\{(\mathbf{a'}\mathbf{y}_s - \mathbf{b'}_a \mathbf{y}_a)(\mathbf{a'}\mathbf{y}_s - \mathbf{b'}_a \mathbf{z}_a)\} - E\{(\bm{\lambda'}\mathbf{y}_s - \mathbf{b'}_a \mathbf{z}_a)(\bm{\lambda}'\mathbf{y}_s - \mathbf{b'}_a \mathbf{z}_a)\}
\end{equation} \noindent is greater than 0 for all \(\mathbf{a'}\). The
prediction equations are

\begin{equation}
\begin{pmatrix}
\bm{\Sigma}_{s, s} & \mathbf{X}_s \\
\mathbf{X}_s' & 0
\end{pmatrix} 
\begin{pmatrix}
\bm{\lambda} \\
m
\end{pmatrix} = 
\begin{pmatrix}
\bm{\Sigma}_{s, s} & \bm{\Sigma}_{s, u} \\
\mathbf{X}_{s}' & \mathbf{X}_{u}'
\end{pmatrix} 
\begin{pmatrix}
\mathbf{b}_{s} \\
\mathbf{b}_{u}
\end{pmatrix},
\end{equation} \noindent where again the subscripts \(s\) and \(u\)
denote sampled and unsampled observations. For example, letting \(n\)
denote the number of sampled observations, \(\bm{\Sigma}_{s, s}\)
denotes the \(n \times n\) submatrix of \(\bm{\Sigma}\) corresponding
only to rows and columns of sampled observations and
\(\bm{\Sigma}_{u, s}\) denotes the \((N - n) \times n\) submatrix of
\(\bm{\Sigma}\) corresponding to rows of observations that were not
sampled and columns of observations that were sampled. Then,
\(\bm{\lambda}\) is an \(n \times 1\) vector.

Then, we can solve for the prediction weights as \mbox{}
\begin{equation}
\bm{\lambda}_s = \mathbf{b}_{s}' + \mathbf{b}_{u}' (\bm{\Sigma}_{u, s}\bm{\Sigma}_{s, s}^{-1}) - \mathbf{b}'_{u}(\bm{\Sigma}_{u, s} \bm{\Sigma}_{s, s}^{-1})\mathbf{X}_s(\mathbf{X}_s'\bm{\Sigma}_{s, s}^{-1}\mathbf{X}_s)^{-1}\mathbf{X}_s'\bm{\Sigma}_{s, s}^{-1} + \mathbf{b}_{u}' \mathbf{X}_{u}'(\mathbf{X}_s'\bm{\Sigma}_{s, s}^{-1}\mathbf{X}_s)^{-1}\mathbf{X}_s \bm{\Sigma}_{s, s}^{-1}.
\end{equation}

Our prediction for the total abundance across all years of the survey is
then \mbox{} \begin{equation}
\bm{\lambda}_s' \mathbf{\tilde{y}}_s,
\end{equation} \noindent with a prediction variance of \mbox{}
\begin{equation}
E((\bm{\lambda}_s'\mathbf{y}_s - \mathbf{b}_a'\mathbf{y}_a)(\bm{\lambda}_s'\mathbf{y}_s - \mathbf{b}_a'\mathbf{y}_a)) = \\
\bm{\lambda}_s'\bm{\Sigma}_{s, s}\bm{\lambda}_s - 2 \mathbf{b}_a' \bm{\Sigma}_{a ,s} \bm{\lambda}_s + \mathbf{b}_a' \bm{\Sigma}\mathbf{b}_a.
\end{equation}

However, we often are not interested in predicting total abundance
across multiple years and instead would want a prediction of the total
abundance in the most recent year of the survey. Therefore, for this
goal, \(\mathbf{b}_a\) should not be a vector of 1's and should instead
take a value of 1 if the observation is in the current year of the
survey and a 0 if the observation is not in the current year of the
survey: \mbox{} \begin{equation}
\mathbf{b}_a = [\mathbf{b}_{up}', \mathbf{b}_{uc}', \mathbf{b}_{sp}', , \mathbf{b}_{sc}']' = [\mathbf{0}', \mathbf{1}', \mathbf{0}', \mathbf{1}']',
\end{equation} \noindent where the subscripts \(up\), \(uc\), \(sp\),
and \(sc\) denote unsampled sites in past years, unsampled sites in
current years, sampled sites in past years, and sampled sites in current
years, respectively.

\(\bm{\lambda}_s\) can then be rewritten as \mbox{} \begin{equation}
\bm{\lambda}_s = \mathbf{b}_{s}' + \mathbf{b}_{uc}' (\bm{\Sigma}_{uc, s}\bm{\Sigma}_{s, s}^{-1}) - \mathbf{b}'_{uc}(\bm{\Sigma}_{uc, s} \bm{\Sigma}_{s, s}^{-1})\mathbf{X}_s(\mathbf{X}_s'\bm{\Sigma}_{s, s}^{-1}\mathbf{X}_s)^{-1}\mathbf{X}_s'\bm{\Sigma}_{s, s}^{-1} + \mathbf{b}_{uc}' \mathbf{X}_{uc}'(\mathbf{X}_s'\bm{\Sigma}_{s, s}^{-1}\mathbf{X}_s)^{-1}\mathbf{X}_s \bm{\Sigma}_{s, s}^{-1}.
\end{equation}

The prediction variance can then be rewritten as \mbox{}
\begin{equation}
\bm{\lambda}_s'\bm{\Sigma}_{s, s}\bm{\lambda}_s - 2 \mathbf{b}_{c}' \bm{\Sigma}_{c, s} \bm{\lambda}_s + \mathbf{b}_{c}' \bm{\Sigma_{c, c}} \mathbf{b}_{c},
\end{equation} \noindent where \(c\) denotes observations in the current
year.

\section{Application}

\section{Simulation}

\section{Discussion}

\hypertarget{tables}{%
\subsection{Tables}\label{tables}}

The \texttt{interact} class file will deal with positioning your tables
in the same way as standard \LaTeX. It should not normally be necessary
to use the optional \texttt{{[}htb{]}} location specifiers of the
\texttt{table} environment in your manuscript; you may, however, find
the \texttt{{[}p{]}} placement option or the \texttt{endfloat} package
useful if a journal insists on the need to separate tables from the
text.

The \texttt{tabular} environment can be used as shown to create tables
with single horizontal rules at the head, foot and elsewhere as
appropriate. The captions appear above the tables in the
\textsf{Interact} style, therefore the \texttt{\textbackslash{}tbl}
command should be used before the body of the table. For example,
Table\textasciitilde{}\ref{sample-table} is produced using the following
commands:

\begin{table}
\tbl{Example of a table showing that its caption is as wide as
 the table itself and justified.}
{\begin{tabular}{lcccccc} \toprule
 & \multicolumn{2}{l}{Type} \\ \cmidrule{2-7}
 Class & One & Two & Three & Four & Five & Six \\ \midrule
 Alpha\textsuperscript{a} & A1 & A2 & A3 & A4 & A5 & A6 \\
 Beta & B2 & B2 & B3 & B4 & B5 & B6 \\
 Gamma & C2 & C2 & C3 & C4 & C5 & C6 \\ \bottomrule
\end{tabular}}
\tabnote{\textsuperscript{a}This footnote shows how to include
 footnotes to a table if required.}
\label{sample-table}
\end{table}

\begin{verbatim}
\begin{table}
\tbl{Example of a table showing that its caption is as wide as
 the table itself and justified.}
{\begin{tabular}{lcccccc} \toprule
 & \multicolumn{2}{l}{Type} \\ \cmidrule{2-7}
 Class & One & Two & Three & Four & Five & Six \\ \midrule
 Alpha\textsuperscript{a} & A1 & A2 & A3 & A4 & A5 & A6 \\
 Beta & B2 & B2 & B3 & B4 & B5 & B6 \\
 Gamma & C2 & C2 & C3 & C4 & C5 & C6 \\ \bottomrule
\end{tabular}}
\tabnote{\textsuperscript{a}This footnote shows how to include
 footnotes to a table if required.}
\label{sample-table}
\end{table}
\end{verbatim}

To ensure that tables are correctly numbered automatically, the
\texttt{\textbackslash{}label} command should be included just before
\texttt{\textbackslash{}end\{table\}}.

The \texttt{\textbackslash{}toprule}, \texttt{\textbackslash{}midrule},
\texttt{\textbackslash{}bottomrule} and
\texttt{\textbackslash{}cmidrule} commands are those used by
\texttt{booktabs.sty}, which is called by the \texttt{interact} class
file and included in the \textsf{Interact} \LaTeX~bundle for
convenience. Tables produced using the standard commands of the
\texttt{tabular} environment are also compatible with the
\texttt{interact} class file.

\hypertarget{the-list-of-references}{%
\subsection{The list of references}\label{the-list-of-references}}

References should be listed at the end of the main text in alphabetical
order by authors' surnames, then chronologically (earliest first). If
references have the same author(s), editor(s), etc., arrange by year of
publication, with undated works at the end. A single-author entry
precedes a multi-author entry that begins with the same name. If the
reference list contains two or more items by the same author(s) in the
same year, add a, b, etc. and list them alphabetically by title.
Successive entries by two or more authors when only the first author is
the same are alphabetized by co-authors' surnames. If a reference has
more than ten named authors, list only the first seven, followed by `et
al.'. If a reference has no author or editor, order by title; if a date
of publication is impossible to find, use `n.d.' in its place.

The following list shows some sample references prepared in the Taylor
\& Francis Chicago author-date style.

\citep{Ade09, Alb05}

\bibliographystyle{tfcad}
\bibliography{interactcadsample.bib}


\input{"appendix.tex"}


\end{document}
